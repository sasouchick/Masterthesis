% !TeX root = ../thesis.tex

\chapter{Introduction}
% Significance: Explain the significance of the research, including why it is important and how it addresses a current gap or need in the field.

The energy sector is undergoing a transition from fossil fuels to renewable energy systems (RES). In 2020, RE technologies %represented ...\% of all the newly installed electricity capacity worldwide and 
accounts for 29\% of global energy capacity~\cite{IEA_2021}.
Simultaneously, the decarbonization and electrification of several sectors, for instance heating and transport, result in additional electricity demand. 

Consequently of the Paris Agreement, Germany has committed itself to reduce greenhouse gas (GHG) emissions by 65\% in 2030 compared to 1990 and to achieve climate neutrality in 2045~\cite{bmwk_klimaschutzpolitik}. The government has adopted a "climate protection program 2022" investing 8 billion euros into sector regarding industry, energy and housing~\cite{bmwk_klimaschutzpolitik}. With decarbonization of these sectors comes a integration of electric mobility and heating. Moreover, the war in Ukraine causes the German government to redefine their relation with Russia, reducing their dependency of natural gas and implementing electric heating options even faster~\cite{ukraine}. 

To curb CO$_2$ emissions from the heating sector, contributing by 18\% for space heating~\cite{agora_wärmewende},
fossil-based gas heating systems must be replaced by electric heat pumps (EHP) with significant higher efficiency in order to reduce the final energy consumption in households by up to 70\%~\cite{Gupta_Pena-Bello_Streicher_Roduner_Farhat_Thöni_Patel_Parra_2021}.
On the 23th of March 2022, the German government decided that every space heating should have a minimum of 65\% of renewable energies by 2024~\cite{Öko-InstitutundFraunhoferISE_2020}.
Regarding electrification of the transport sector, contributing to 23\% of the carbon emissions~\cite{co2transport}
Germany wants to become the leading sector in electric mobility by introducing 15 million battery electric vehicles (BEV) with one million public charging points until 2030~\cite{Bmwk_eautos}. Moreover, the "governing program electric mobility" aims to increase buyers attractiveness by building exclusive e-parking spots, special lanes but also reducing taxes and introducing bonuses for sustainable cars~\cite{Bmwk_eautos}.

Henceforth, it essential to increase the expansion of the German distribution grid in the interest of the development of the electricity consumption at end-consumer level. The introduction of heat pumps (HP) and EVs pressingly modifies the load of the connected buildings. Peak electricity demand will raise~\cite{ior}. As of now these two technologies have been integrated into the network, however with limited impact on the grid infrastructure. The current hosting capacity will be exceeded and a need for grid improvement or partial rebuild will be necessary~\cite{Ismael_AbdelAleem_Abdelaziz_Zobaa_2019}.
The deployment of HP and EV must be planned carefully to guarantee a smooth and cost-effective energy transition for distribution system operators (DSO) and final consumers. 


\section{State-of-the-Art}

 In the UK, a study analysing suburban low voltage networks has developed a probabilistic model in order to assess the impacts of EVs and HP \cite{navarro-espinosa_probabilistic_2014}. This study concluded that thermal issues increased more often and at much earlier penetration level than voltage instability. Additionally, thermal issues were more likely moving from the upstream components, such as transformers, to the downstream equipment's. Regarding the integration of HPs, the penetration level for ground source heat pumps was much higher than for air source heat pumps until thermal overloading occurred. However, this study has several uncertainties, for instance the exact location of the HPs or reactive power consumption. In this thesis, the the simulation is done on an existing grid, meaning that exact locations of the new loads is known and those loads are adapted to the individual households and buildings of the area. 

In \cite{GUPTA2021116504}, the scenarios of integration by EV, PV and HP are analysed in Switzerland. Three to four scenarios are analysed for their penetration in the years 2035 and 2050. Similar to this thesis, a GIS-based method is developed in order to map technical potentials of the area. This research is also located in a low voltage grid. It was found that, PVs cause more voltage violations compared to HPs and EVs with 18.5\% and 13.7\% respectively. Additionally, EVs and HPs cause on the other hand higher line overloading than PVs. Regarding grid reinforcement, the study concluded that the option of battery energy storage system could reduce costs up to 15\% compared to exchanging a higher rating transformer stations. However, this research is first of all not done in Germany, which might have a different grid structure. Especially regarding the scenarios, a study done in Germany is important in order to set the scenario goals equivalent to the political goals of the government. Additionally, it has been analysed in an urban area and was then upscaled to national level. investigations established that the cost of grid reinforcement were considerably higher in rural areas compared to suburban or urban areas. The cost increase lead to 410\% with HPs and 90\% with EVs. While this thesis focuses on the outcomes of a rural area, an upscaling from urban to national wide areas might be faulty. Usually, a rural area has less energy density and less hosting capacity than an urban grid~\cite{ior}. Hence, the necessary future grid reinforcements might be different in depending the area. 

A study from the Netherlands, \cite{VELDMAN2013233}, aims to investigate the impact of various future scenarios on the electricity distribution networks in the Netherlands, with a focus on residential areas. The need for a transition to a more sustainable energy system is acknowledged, but the pace of this transition is not fast enough to achieve policy goals. Three scenarios are created based on different drivers, such as economic development, climate policy, and international cooperation. The impact of new technologies, such as micro combined heat and power systems, heat pumps, photovoltaic panels, and electric vehicles, on the distribution grids is analysed. The scenarios are further used to assess the expected demand growth of electricity, the voltage level to which generators are connected, and the development of new technologies applied at the residential level. Flexible electricity demand is also discussed, and its potential advantages, such as leveraging loads without expensive storage technologies and reducing energy loss due to the transportation of electricity. However, conflicting objectives between different parties, such as transmission system operators, commercial energy suppliers, and network operators, may result in non-optimal shifting of electricity demands. The impact of various control strategies on the grids is therefore considered an important step towards the development of smart grids. The project does not consider the impact of policies or technologies outside of the Netherlands or other external factors, such as changes in energy markets or geopolitical events. It could go more in-depth on the specific challenges and opportunities for grid management and the potential benefits and drawbacks of different load management strategies. In case of grid reinforcement and flexibility, \cite{VELDMAN2013233} states that load peaks can be reduced by 35–67\% in various scenarios in the medium voltage grid in the Netherlands. Load management in future cases would be a financially effective method, reducing costs by 45–72\% in investment~\cite{VELDMAN2013233}. Load management is considered in different sectors in this study. Flexible loads are partly shifted to reduce peak demands, including charging EV at night, using a buffering heat storage for HPs or increasing smart appliances that activate when demand is low. This study is, however only focusing on medium voltage grids, whereas the flexibility management in low voltage grids might be trickier, due to smaller but more non-adjustable factors interacting~\cite{ior}. 

On a more recent note, paper \cite{ARNAUDO2020117012} focuses on the energy infrastructure in Stockholm, specifically the issue of the saturation of the electricity distribution grid capacity due to the increasing use of heat pumps (HPs) in residential multi-apartment buildings. While the majority of these buildings are currently connected to a district heating (DH) network, some neighbourhoods are planning to switch to distributed HP-based heating infrastructures in order to achieve economic savings. The paper proposes demand side management (DSM) solutions to integrate distributed HPs, using indoor thermostat control devices and the buildings' thermal mass as energy storage. A techno-economic analysis is presented to assess the feasibility of these solutions, and a co-simulation approach is used to couple the electricity grid and buildings through a feedback control, monitoring the grid capacity and controlling the indoor temperature. The paper builds on previous studies by integrating specific models for the performance simulation of networks and buildings' thermal balance, and using a power flow analysis tool (Pandapower) and a two-RC thermal demand model. The case study focuses on the Hammarby Sjöstad neighbourhood in Stockholm, and explores the potential of using distributed thermal energy storage units and thermal mass control in buildings to avoid overloading the grid. The paper presents the case study area and introduces the BBs (building blocks), which are aggregations of buildings identified based on their heated area. The existing heating and electricity installed capacities are estimated around 1.2 MW and 1.6 MW, respectively, and the heating demand is currently covered by a DH network while the electricity demand is balanced by an electricity distribution grid. Eventually, the study found that the infrastructure should still be partly connected to the district heating (DH) network (around 7\% of the heat demand) given the grid’s capacity limits. However, this dependency can decrease by around 1\% when the buildings’ thermal mass is used as thermal storage, with a range of $\pm$ 0.5 $\circ C$. On a heat pump level, the disconnections decrease up to 50\%, depending on the buildings’ thermal mass capacity. This leads to better techno-economic and environmental performances. The study explores the potential of using the buildings’ thermal mass as energy storage to unlock the integration of distributed heat pumps and proposes demand-side management solutions to avoid grid overloading. However, for neighbourhoods with mainly gas as its heat infrastructure without being connected to a district heating network, the findings of this study may not be directly applicable. The impact of distributed heat pumps on an electricity distribution grid and the potential for using thermal mass as energy storage may not be relevant. However, the approach presented in this paper could still be of interest to neighbourhoods with gas-based heating systems that are considering transitioning to distributed heat pumps. The study provides a framework for analysing the impact of such a transition on the electricity distribution grid and highlights the importance of demand-side management solutions to avoid grid overloading. 

Moreover due to the recent war in Ukraine, most studies have slower electrification scenarios compared to what is appropriate today. The following thesis is basing its scenario scope around the published predictions from Bundesnetzagentur of Germany regarding the increase of EVs and HP~\cite{netzentwicklungsplan}. The "Netzentwicklungsplan 2023" orients itself on the current political and technical issues in order to forecast appropriate scenario forecasts assuming achievable goals and possible outcomes. The final goal of the plan is to achieve climate neutrality by 2045~\cite{netzentwicklungsplan}. This project is the most current prediction that distribution grid operators will orient themselves on and therefore this thesis will be one of the first to apply these scenarios on a low voltage grid. Compared to other researches, the future thesis will focus on a grid in a rural area, where behaviour and energy distribution vary from urban areas. The influence of EVs and HPs on a distribution grid has been researched before, however, the combination of both in rural areas and in a real life grid is still unclear. Moreover the low voltage grid in question mainly consists of overhead-lines which have lower hosting capacity than underground cables. In rural areas, especially tourist resorts, this is the norm~\cite{ior}. Hence the study might be representable for other similar regions. 

%maybe interesting? https://www.cired-repository.org/bitstream/handle/20.500.12455/302/CIRED%202019%20-%201118.pdf?sequence=1&isAllowed=y

\section{Goal and Scope}
% Research questions: Clearly state the research questions that the thesis aims to answer.
The goal of the master thesis is to analyse the impact of e-mobility and heat pump integration on a real low voltage grid in terms of voltage stability and loading equipment. The scope of the study is limited to the load of electric vehicles (BEVs) and heat pumps, with a focus on home charging and work charging. The thesis will also examine the effects of increasing load on the grid and identify potential weaknesses in the electricity grid. The study will involve simulation models in PowerFactory to explore different scenarios and variants. The objectives of the thesis are to identify short- and long-term construction measures to ensure the functionality and reliability of the network and to provide proposals for measures to improve the network. The study will not consider the impact of increasing households or demographic changes in the area, and will not include photovoltaic in-feed. Additionally, during the grid reinforcement it is considered that all equipment is strong enough, so that no fault occurs during short circuit calculations in the future scenarios.

The analysed grid in this thesis has a meshed grid structure in a rural area of western Germany, consisting mostly of overhead lines and serving approximately 315 buildings. The grid features three two winding transformers, and is designed to distribute electricity to the attached buildings. The analysis of the impact of e-mobility and heat pump integration on this grid will provide insight into the challenges and opportunities posed by these developments in a rural area, and will inform the design and implementation of future low voltage grids. By examining the effects of increasing load on the grid and identifying potential weaknesses, the thesis will contribute to the development of a more resilient and sustainable electricity grid. Eventually this study will help identify potential solutions for managing increasing loads on the grid, including reinforcement or expansion of equipment, optimisation of grid management and operation, and hence assess the economic effect by comparing the cost-effectiveness of different solutions for the distribution system operator. 

%Limitations: Discuss any limitations of the study, including any data or information that will not be considered, any assumptions that have been made, or any limitations in the methodology.
On the other hand, there are several limitations that should be taken into account. Firstly, the study is solely focused on the specific grid being analyzed, and the solution may differ for other grids. Secondly, conventional grid reinforcement is mainly considered with small amounts of flexibility. Thirdly, photovoltaic (PV) technology is not included in the study since it mainly focuses on load impact during the winter months when the sun is not shining. Fourthly, the study assumes one car per household based on the Netzentwicklungsplans scenario developments. Additionally, the scenarios used in the study are based on current knowledge, and any future policy changes may affect their accuracy. EV charging is limited to 11 kW per household since anything higher must be recognized by the grid operator. Public charging stations will not be considered since the analyzed area has low charging potential due to being a rural area with few travelers. Household energy consumption data is based on statistics regarding the level of renovation and residents in Germany due to limited data availability. Finally, mobility related to EV charging is based on statistical data from Germany on duration, destination, arrival, and the number of trips during the day.


% Timeframe: Provide a timeframe for the completion of the thesis, including any milestones or key deadlines.
% Contributions: State how the thesis will contribute to the field, including the expected outcomes, implications, and impact of the research.
