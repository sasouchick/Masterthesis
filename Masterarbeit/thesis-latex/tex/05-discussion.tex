\chapter{Discussion}



%wp las profile
with climate change, the ambient temperature will be warmer and warmer, less need for high heat pumps
heat pump technology will improve fast and become more efficient

%auto last profile
verkehrswende kann dazu führen, dass weniger e autos am netz hängen als geplant, weil ÖPNV verbessert wurde ( obwohl im ländlichem Gebiet unwahrscheinlich) oder car sharing

immediate strategy: This mimics a setting where drivers have no incentives and/or no technical possibility to charge their vehicle batteries in a more balanced way, which is likely to be sub-optimal with respect to the electricity market or network situation.

%Load Flow 
The results from the studies, whose primary objective was to perform a strategic impact analysis with indications of the main bottlenecks and trends, indicate that thermal problems are likely to arise at much earlier penetration levels than for voltage problems, and moving from upstream components (transformers) to downstream ones (feeders and starting branches of laterals). This is essentially due to the much less diversity present in heating loads which, when electrified, brings issues in those part of networks designed for higher diversity. While the results cannot be generalised to all situations, they give key indications about the adequacy of the network in the long term, pointing out when and where problems could arise (in probabilistic terms) and the drivers for impact, so that network replacement or other solutions can be put in place.

%HP tech differences
 It is finally interesting to remark that for the base case and for all sensitivity analyses the problems appear for earlier penetration levels in the ASHP case relative to the GSHP case. For example, in the base case the thermal problems start at 50\% penetration level for the GSHP and at 40\% penetration level for the ASHP. Also, in the GSHP case the voltage problems almost disappear for every penetration level and the daily energy losses are significantly less than in the ASHP case, particularly for higher penetration levels (as losses are quadratic with respect to the load). These results are essentially due to the better performance of GSHPs that exploit higher temperature sources, which is particularly noticeable in the “coldest day” simulations that have been performed. Hence, GSHPs do not only generally prove to be more efficient in terms of energy performance, but also feature less network impact.

%regarding grid investments
From the network operator's perspective, even though the cost-optimal perspective is preferable, network operators have no financial incentive to invest into grid equipment with high operational costs. Operational costs are generally disadvantaged due to the fixed interest on assets. The regulatory system offers here through interim profits in the current regulation period, short-term profit opportunities, but no long-term incentives. The system should therefore be fundamentally revised so that
these measures also appear worthwhile from the point of view of the actors.~\cite{mona}